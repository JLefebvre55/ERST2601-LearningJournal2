\documentclass{report}
\usepackage{setspace} % Setting line spacing
\usepackage[normalem]{ulem} % Underline
\usepackage{caption} % Captioning figures
\usepackage{subcaption} % Subfigures
\usepackage{geometry} % Page layout
\usepackage{multicol} % Columned pages
\usepackage{array,etoolbox}
\usepackage{fancyhdr}
\usepackage{enumitem}
\usepackage[table]{xcolor}
\usepackage[toc,page]{appendix}
\usepackage{titlesec} % Section formatting

\usepackage[backend=biber,style=apa,citestyle=authoryear]{biblatex}
\DeclareLanguageMapping{english}{english-apa}
\DeclareFieldFormat{journaltitle}{\textit{#1}}
\DeclareFieldFormat[article]{volume}{\textit{#1}}
\DeclareFieldFormat[misc]{title}{\textit{#1}}
\DeclareFieldFormat[inbook]{booktitle}{\textit{#1}}
\DeclareFieldFormat[book]{title}{\textit{#1}}
\renewcommand*{\nameyeardelim}{\addcomma\space}
\DeclareDelimFormat[parencite]{finalnamedelim}{\addspace\&\space} % Use `&' instead of `and' in citations
\addbibresource{references.bib}

\titleformat{\section}{\normalfont\fontsize{12}{15}\bfseries}{\thesection}{1em}{}
\titleformat{\subsection}{\normalfont\fontsize{12}{15}\bfseries}{\thesubsection}{1em}{}

\AtBeginEnvironment{quote}{\par\onehalfspacing\small} % Block quotes

% Page layout (margins, size, line spacing)
\geometry{letterpaper, left=1in, right=1in, bottom=1in, top=1in}
\setstretch{2}

% Headers
\pagestyle{fancy}
\lhead{ERST2601 Learning Journal II}
\rhead{Jayden Lefebvre}
\setlength{\headheight}{16pt}

\begin{document}

\begin{titlepage}
    \begin{center}
        \vspace*{1.2cm}

        \textbf{Learning Journal II}

        \vspace{2cm}

        Jayden Lefebvre\\

        \vspace{5cm}
        
        Trent University\\
        ERST 2601Y 2024WI\\
        Dr. Dan Longboat\\

        \vfill

        March 31st, 2025
        
    \end{center}
\end{titlepage}

\thispagestyle{plain}
\tableofcontents

\clearpage

\section{On Storytelling}

\hspace{24pt} Oral traditions of knowledge sharing are a powerful tool for the embedding of cultural values and moral teachings into a single woven narrative.

% Jan 6th & 13th

% Origins and transmission of knowledge

% Life lessons, natural world

% Dreams, visions, prophecy

% Creation teachings
% Original Instructions
% 1. Love, Compassion, Help/Be of Service
% 2. Live within cycles/balance of nature, learn from nature
% 3. Be Thankful of All Elements of Creation

% 

\clearpage

\section{On Ceremonies}

\hspace{24pt} As somebody who grew up in a Western secular context, I have never been truly exposed to ceremonies. I have come to understand through my learning as part of this course that ceremonies are more than they appear: they exist at the intersection of the physical with the spiritual, and delineate the threshold between the past and the future.

\hspace{24pt} % TODO: Purpose of ceremonies and teachings

\hspace{24pt} % TODO: Consequences of disconnection from ceremonial life

% Four Sacred Ceremonies:
% Great Feather Dance
% Men's Thanksgiving Song
% Drum Dance
% Peach Stone Game - nature in harmony

% Ceremonies as a uniting framework
% Forest's edge protocol

% Without ceremonies, people become selfish, afraid, and disconnected

% As simple as giving thanks for the food we eat, burning tobacco, shaking hands

% People disconnected become destructive and adversarial toward nature, one another, and themselves

% Process of Forgiveness: Naming the wrong, expressing why it was done, resolving never to do it again

\clearpage

\section{On Yellowknife Dene First Nation}

\hspace{24pt} On February 3rd, our class was priviledged to participate in an interview session with retired Chief of Yellowknife's Dene First Nation (the YKDFN), Jonas Sangris. This discussion explored themes related to politics, ecosystem degradation, and the loss of traditional ways of life.

\hspace{24pt} Through this discussion, I was inspired to focus my final research paper on the YKDFN and their traditional foodways. The interview explored the impacts of industrial activity on the YKDFN's traditional territories and the subsequent endangerment of foodways, especially concerning caribou herd migration and the influence of arsenic contamination due to mining. This oral source provided a much-needed firsthand perspective on the issues facing the YKDFN.

\clearpage

\section{On Plant Medicines}

% Joseph Pitawanakwat - Creator's Garden

\hspace{24pt} On March 10th, our class hosted guest speaker Joseph Pitawanakwat, founder of Creator's Garden, who delivered a lecture on traditional Indigenous knowledge around plant medicines.
Joe's presentation began with stories from his personal upbringing, including stories about his grandmother and her relationship to plants and her inherited knowledge on the applications of plants as medicines in a traditional context. 
Joseph continued by explaining his own journey within academia and his work delivering medicines to communities in need.

\hspace{24pt} Firstly, I was impacted by the potential for plant medicines to be used in treating conventionally incurable ailments, such as arthritis. Coming from a Western scientific background, these statements appeared contradictory to my understanding of medicine, yet Joseph's personal stories of the multitude of people he has helped were convincing.

\hspace{24pt} Additionally, Joseph's story of how his findings were rejected by Western medicine was a heartbreaking reminder of the systemic institutional barriers to healing that all people face. I personally reflected on this as representative of the broader issue of the recognition of Indigenous knowledge in academia and the value of lived experience in the scientific process.

\hspace{24pt} Finally, I found it interesting how each plant species' individual physiology reflected its use as a medicine. For example, a plant whose stem ressembled a bone is used to treat osteoporosis to great effect. This once again challenged my understanding of the world and medicine.

\clearpage

\section{On Treaties}

% Great Law of Peace: Peace, Unity, and a Good Mind (as opposed to guilt, shame, fear, anger, etc.)

\clearpage

\section{On Sustainability}

% Old vs new thinking, old vs new technologies

% Global warming

\clearpage

\section{On Anishnaabe Life}

\hspace{24pt} On two occasions, we have had the priviledge of hearing from Jack Hogarth.

\clearpage

\section{On Haudenosaunee Life}

% Clans and kinship; grief, loss, and healing

\clearpage

% References
\printbibliography[heading=bibintoc]

\end{document}