\documentclass{report}
\usepackage{setspace} % Setting line spacing
\usepackage[normalem]{ulem} % Underline
\usepackage{caption} % Captioning figures
\usepackage{subcaption} % Subfigures
\usepackage{geometry} % Page layout
\usepackage{multicol} % Columned pages
\usepackage{array,etoolbox}
\usepackage{fancyhdr}
\usepackage{enumitem}
\usepackage[table]{xcolor}
\usepackage[toc,page]{appendix}
\usepackage{titlesec} % Section formatting

\usepackage[backend=biber,style=apa,citestyle=authoryear]{biblatex}
\DeclareLanguageMapping{english}{english-apa}
\DeclareFieldFormat{journaltitle}{\textit{#1}}
\DeclareFieldFormat[article]{volume}{\textit{#1}}
\DeclareFieldFormat[misc]{title}{\textit{#1}}
\DeclareFieldFormat[inbook]{booktitle}{\textit{#1}}
\DeclareFieldFormat[book]{title}{\textit{#1}}
\renewcommand*{\nameyeardelim}{\addcomma\space}
\DeclareDelimFormat[parencite]{finalnamedelim}{\addspace\&\space} % Use `&' instead of `and' in citations
\addbibresource{references.bib}

\titleformat{\section}{\normalfont\fontsize{12}{15}\bfseries}{\thesection}{1em}{}
\titleformat{\subsection}{\normalfont\fontsize{12}{15}\bfseries}{\thesubsection}{1em}{}

\AtBeginEnvironment{quote}{\par\onehalfspacing\small} % Block quotes

% Page layout (margins, size, line spacing)
\geometry{letterpaper, left=1in, right=1in, bottom=1in, top=1in}
\setstretch{2}

% Headers
\pagestyle{fancy}
\lhead{ERST2601 Learning Journal II}
\rhead{Jayden Lefebvre}
\setlength{\headheight}{16pt}

\begin{document}

\begin{titlepage}
    \begin{center}
        \vspace*{1.2cm}

        \textbf{Learning Journal II}

        \vspace{2cm}

        Jayden Lefebvre\\

        \vspace{5cm}
        
        Trent University\\
        ERST 2601Y 2024WI\\
        Dr. Dan Longboat\\

        \vfill

        March 31st, 2025
        
    \end{center}
\end{titlepage}

\thispagestyle{plain}
\tableofcontents

\clearpage

\section{On Storytelling}

\hspace{24pt} Oral traditions of knowledge-sharing are a powerful tool for the embedding of cultural values and moral teachings into a single woven narrative. On many occasions, I have personally remarked that course lectures are more like sermons; rather than present the teachings in a heirarchical Western format, the lectures are delivered as stories with lessons nested within. As a science student, this is something I am unaccustomed to, but I have found it to be a more engaging and effective method of learning certain teachings. It demands a deeper level of engagement and reflection, and if I do not give the presenter my utmost attention, I tend to miss details that are critical to extracting the lessons.

\hspace{24pt} One of the most impactful aspects of oral tradition, to me, is that the origins of a lesson are contained within the vessel through which the lesson is delivered. For example, the story of the Peacemaker is not just a story about peacekeeping, but a story that \textit{is} peacemaking. The story preserves the law by preserving its \textit{origins} in a narrative format; rather than needing to be enforced, it is integral to the culture. This is a stark contrast to the Western legal system, where laws are enforced by the state and are often disconnected from their origins, leading to a lack of respect for the law and controversy around the legitimacy of its enforcement.

\hspace{24pt} Revealed knowledge is also an important part of many stories. Visions, dreams, and prophecies are preserved and inherited for generations, and provide a centerpoint around which the narrative revolves. Rather than rely solely on the material world to provide guidance, as is done in Western science, Indigenous knowledge has the opportunity to derive wisdom from spiritual practices and their intersection with the natural world. This is a powerful tool for understanding the world and our place within it, and is something that I have not personally encountered before taking this course.

% Jan 6th & 13th

% Origins and transmission of knowledge

% Life lessons, natural world

% Dreams, visions, prophecy

% Creation teachings
% Original Instructions
% 1. Love, Compassion, Help/Be of Service
% 2. Live within cycles/balance of nature, learn from nature
% 3. Be Thankful of All Elements of Creation

% The Peacemaker: The story preserves the law by preserving its origins in a narrative format; Rather than needing to be enforced, it is integral to culture

% . 

\section{On Ceremony}

\hspace{24pt} As somebody who grew up in a Western secular context, I have never been truly exposed to ceremonies. I have come to understand through my learning as part of this course that ceremonies are more than they appear: they exist at the intersection of the physical with the spiritual, and delineate the threshold between the past and the future.

\hspace{24pt} Ceremonial life unites people through shared experience and is foundational to maintaining culture and tradition. The most fundamental aspect of ceremony includes the protocols of everyday living, including giving thanks to one another and the land, the sharing of food, and the gifting of tobacco. One of my favourite examples of this is the "forest's edge" custom, where visitors to a village stop at the edge of a clearing and light a smokey fire to signal their arrival, rather than barging in unwelcomed. This maintains a sense of mutual respect between neighbouring peoples, upholds lasting peace, and fosters a broader sense of community.

\hspace{24pt} Without ceremony, people become self-centered, forgoing gratefulness and respect for the land that provides everything they need to live. This disconnection leads to fear, where people lack a sense of belonging to something greater than themselves, and become as adversaries to nature, to one another, and to themselves. The most foundational ceremonial teaching, in my view, is the process by which we forgive one another: by naming the wrong, expressing why it was done, and resolving never to do it again, we rebuild the trust that precedes all relationships.

\section{On Treaties}

% Treaties we have with the creator
\hspace{24pt} Treaties are living agreements and not only include tactical documents that outline the rules, terms, and conditions of a relationship, but include teachings of mutual devotion to the creator, the people, and the land. For example, in the Haudenosaunee creation teachings, the ``original instructions'' serve as the basis for a treaty between the people and all of creation. So long as people adhere to the principles of love, service, and gratitude, and live within the balance of the cycles of nature, creation will continue to provide for the human race. If the human beings fail to uphold the agreement, the land will cease to provide for them.

% Creation teachings
% Original Instructions
% 1. Love, Compassion, Help/Be of Service
% 2. Live within cycles/balance of nature, learn from nature
% 3. Be Thankful of All Elements of Creation
% Or else nature will cease to provide for us

% Treaties we have with each other
\hspace{24pt} A more conventional and familiar example of a treaty is the Great Law of Peace, which serves as the foundational document of confederation for the six Haudenosaunee nations. As the Peacemaker travelled to influence the warring nations, he evangelized three fundamental teachings: Peace, as opposed to war; Unity, as opposed to division; and a Good Mind, as opposed to guilt, shame, fear, and anger. In bringing together the fifty chiefs, the Peacemaker established a brotherhood of nations that established enduring peace for centuries, symbolized by the evergreen and ever-growing white pine tree - the tree of peace.

% Great Law of Peace: Peace, Unity, and a Good Mind (as opposed to guilt, shame, fear, anger, etc.)
% Bringing together the nations, travelling to influence the warlords
% Brothership of nations, fifty chiefs

% White Pine: The tree of peace is evergreen and evergrowing

% Treaties we have with the land
\hspace{24pt} One area I would like to explore is our treaties with the land. The Earth is a living being, as much as (if not more than) each human being, and as such, each human being has a responsibility to care for it as we would one another. I am reminded of the ``dish with one spoon'', where the Mother Earth is a bowl of beaver tail soup, and each person eats one after the other with a single spoon; satisfy yourself, but leave enough for everyone else - and future generations. This exists as a metaphor for sustainable living with the land, and serves as a sort of ``baseline'' for treaties we might have with the land. Institutional limits to human consumption, such as hunting tags, fishing licenses, and forestry regulations, while insufficient in addressing the root causes of environmental degradation, may prove to be a useful starting point for the development of treaties with the land into the remainder of the 21st century and beyond.

% Dish with one spoon: Beaver tail soup, bowl (mother earth), one spoon; shared resources
% Satisfy yourself but leave enough for everybody else - and future generations

% In the same way that the people require treaties to liberate themselves from institutional tyranny, the bioculture needs treaties to liberate itself from individual tyranny.

\section{On Yellowknife Dene First Nation}

\hspace{24pt} On February 3rd, our class was priviledged to participate in an interview session with retired Chief of Yellowknife's Dene First Nation (the YKDFN), Jonas Sangris. This discussion explored themes related to politics, ecosystem degradation, and the loss of traditional ways of life.

\hspace{24pt} Through this discussion, I was inspired to focus my final research paper on the YKDFN and their traditional foodways. The interview explored the impacts of industrial activity on the YKDFN's traditional territories and the subsequent endangerment of foodways, especially concerning caribou herd migration and the influence of arsenic contamination due to mining. This oral source provided a much-needed firsthand perspective on the issues facing the YKDFN.

\hspace{24pt} Prior to learning about the YKDFN, I was unaware of the plight facing the Indigenous Peoples of Canada's north. Food insecurity is a real everyday issue for the Dene people, and mismanagement by government is only worsening the problem. The YKDFN's struggle to persist in traditional ways of life is emblematic of broader issues surrounding anthropogenic climate change and the exploitation and degradation of the land for the sake of resources and profit, and if we do not find a way to address it - in a way that respects the land and the sovereignty of the people that live in place on it - we stand to lose the systems of knowledge associated with the traditional foodways of the YKDFN.

\section{On Sustainability}

% Old vs new thinking, old vs new technologies

% Global warming

\hspace{24pt} Often during seminar discussions we are confronted with the nature of an unsustainable and exploitative relationship with the land that is characteristic of Western capitalism. Between exponential population growth, rapid industrialization and commodification of natural resources, and contamination and collapse of ecosystems, we have explored through both lecture and seminar many factors that have led to the current state of the world. At the outset, this paints a bleak picture for the future of the human race and the planet Earth, however from these topics have sprung forth conversations about the power of living in place, new ways of thinking, and appropriate technologies.

\hspace{24pt} If we are to confront the challenges of the 21st century and beyond, several key aspects must be addressed. First and foremost among these is our relationship to the natural world. It was stated in one seminar that there is now more human-extracted mass on Earth than there is living biomass. This is both alarming and irreversible, and the only way to counteract the effects of this is to focus our energies on increasing the health and wellbeing of the ecosystems that remain. Secondly, we must consider the role of old ways of thinking in the exacerbation of these problems. If the institutions and systems that we work for and report to do not serve the interests of the human race and the natural world, we must find ways to circumvent them. The concept of "living in place" is one that has tactical applications in this domain; by focussing each person's potential to what is local, we can begin to construct decentralized systems of food production, economy, and governance that will ultimately most closely reflect those of Indigenous Peoples.

\hspace{24pt} It is through new thinking and new technology that these changes will come about. Between renewable energy technology and sustainable agriculture, we ought to be able to develop systems that begin to regenerate the ecosystems that have been degraded, while simultaneously decreasing our reliance on centralized and top-down systems of production and governance. Furthermore, these localized systems will pave a path toward more resilient communities and away from the traumatic processes of colonization.

\section{On Anishinaabe Spirituality}

\hspace{24pt} On two occasions, we have had the privilege of hearing from Jack Hoggarth, Assistant Professor and Chair of Anishinaabe Knowledge within the Chanie Wenjack School for Indigenous Studies at Trent University. On the first occasion, Jack taught us about the Anishinaabe creation story, and on the second, he shared with us the Anishinaabe calendar and the meaning and significance of each moon and the seasons, as well as certain spiritual teachings.

\hspace{24pt} Firstly, I am fascinated by the 13-moon calendar system, which more accurately reflects the movements of the Earth and the cycles of nature than the Western Gregorian calendar. The year begins in May, when trees begin to bud; this is also a time of fasting. Each moon is emblematic of a particular activity: Some moons are for harvesting berries, others designate hunting or Maple sap collection, and others are for certain spiritual activities, including fasting, storytelling, and the making of offerings. The moon system of timekeeping and seasonality is reflective of a deeper understanding of the experience of time as something that is cyclical and recombinant, rather than linear and progressive. To my understanding, each year is as important as the last, and presents yet another opportunity to connect with the Earth and ourselves.

\hspace{24pt} Jack also spoke about an understanding of human spirit with five layers: the spirit, which we are born knowing; the ghost/shadow of the spirit, which is cleansed through ceremony; the mind, which is masculine; the body and heart, which is feminine; and the ``aura'', an outward projection of light. According to Anishinaabe teachings, experiences of trauma, abuse, and shame will form a ``bark'' around your being that prevents the spirit from working through you. Jack spoke about being ``guided by spirit'', which reminded me of a Christian teaching from my youth: ``The hand of God will never lead you where the grace of God will not protect you''.


\section{On Haudenosaunee Life}

\hspace{24pt} Of the many aspects of Haudenosaunee life that have been explored through this course, clans and kinship have stood out to me as being the most absent from Western society. The idea of community built on family that extends beyond blood leads to the formation of a social web wherein each member can be supported by multitudes of peers and elders, and can feel an immediate brother/sisterhood with individuals they have just met. This is directly opposed to the individualistic notions of nuclear family that are standard in Western society, where each member of a family has only a few siblings and one or two parents to rely on. One major instance where this matters most is in times of grief and loss: when a member of a clan passes away, the entire clan is there to support the grieving family, and the other clans are there to support through the grieving process.

\hspace{24pt} Yet another uniting force in the formation of the Haudenosaunee social web is the unique structure of greetings and introductions between villages. 
The greeting begins with a story on how the introduction came to be: our elders  and clan mothers spoke about this meeting, and we bring their love and gratitude to yours. It progresses then to assurance that we take care of ourselves and ensure the thriving of future generations, and encourage the same for you.
Then, one by one, each demographic is represented and their purpose stated: our faith-keepers and elders are still guiding, our men and women still work hard and take care of each other and children, and our children continue to learn and explore. Finally, the greeting ends with a statement: so sure are we that our respective nations will continue, that even our not-yet-born love your not-yet-born.
By embedding the roles of the entire community in the greeting, this process serves as the foundation for all future interactions, and ensures that all parties are of one mind prior to engaging in any business or treaty-making.

% Greeting other nations of people: Diplomacy, solidify each other AND relationship, Of One Mind before any business or treaty-making

% 1. Greetings from my nation
%     1. Elders and clan mothers and all met and spoke about coming here
%     2. Bring our love, gratitude, appreciation to other clan mothers
%     3. WE TAKE CARE OF OURSELVES AND FUTURE GENERATIONS AS BEST WE CAN
%     4. Encourage the same for you
% 2. Faith-keepers, elders: still guiding
% 3. men: still working hard and taking care
% 4. women: still preparing food and raising children
% 5. children: still learning and exploring
% 6. And so sure are we that our respective nations will continue, that even our not-yet-born love your not-yet-born

% > Roles of all people embedded in the greetings.

\section{On Plant Medicines}

% Joseph Pitawanakwat - Creator's Garden

\hspace{24pt} On March 10th, our class hosted guest speaker Joseph Pitawanakwat, founder of Creator's Garden, who delivered a lecture on traditional Indigenous knowledge around plant medicines.
Joe's presentation began with stories from his personal upbringing, including stories about his grandmother and her relationship to plants and her inherited knowledge on the applications of plants as medicines in a traditional context. 
Joseph continued by explaining his own journey within academia and his work delivering medicines to communities in need.

\hspace{24pt} Firstly, I was impacted by the potential for plant medicines to be used in treating conventionally incurable ailments, such as arthritis. Coming from a Western scientific background, these statements appeared contradictory to my understanding of medicine, yet Joseph's personal stories of the multitude of people he has helped were convincing.
Additionally, Joseph's story of how his findings were rejected by Western medicine was a heartbreaking reminder of the systemic institutional barriers to healing that all people face. I personally reflected on this as representative of the broader issue of the recognition of Indigenous knowledge in academia and the value of lived experience in the scientific process.
Finally, I found it interesting how each plant species' individual physiology reflected its use as a medicine. For example, a plant whose stem ressembled a bone is used to treat osteoporosis to great effect. This once again challenged my understanding of the world and medicine.

\hspace{24pt} Given the potential for plant medicines to treat previously incurable illnesses, I believe that it is crucial to deliver these treatments to the people who need them. I was left wondering what barriers exist to the cultivation and production of these plants; if grown outside their natural habitat, would they still retain their medicinal properties?

\section{On My Digital Story}

\hspace{24pt} One of the assignments this semester was to create a digital story, and along with it, a critical reflection. I chose to focus on my background as a scientist and engineer, and how my learning in this course has influenced me to realign my perspective on problem-solving with the lived experiences of the stakeholders involved.

\hspace{24pt} When I was confronted with a design task as an engineer, the standard protocol was to first evaluate the requirements of the project. By establishing a set of solution objectives and breaking down the problem into measurable metrics of success, I could be sure that, once I had developed a solution, I could validate that it met the needs of the problem and verify the success of a prototype implementation. While this process did incorporate stakeholder's values and feedback as part of the iterative process, the top-down nature of the design process often led to a disconnect between the solution and the lived experience of the people who would be using it.

\hspace{24pt} One of the examples I used in my digital story was the design of an electronic device to assist the visually impaired in identifying articles of clothing in a wardrobe. The device did successfully solve the problem: by attaching wireless tags to each piece of clothing and associating these tags with a recorded audio message, the device could identify and play back the associated voice memo to the user. However, the solution was incomplete: the device was over-engineered and difficult to use, rendering the problem only partially solved. While the solution met my needs as a designer, incorporating elements from technology that interested me, it did not help anyone but myself.


\section{On This Learning Journal}

\hspace{24pt} When I came to Trent University to study environmental science and sustainable agriculture, I did so with the intention of learning to solve problems using scientific and technological means. I wanted to apply my skillset as an engineer to make it possible to live more sustainably and in harmony with the natural world. I have since come to realize, through my experience in this course, that, if we are to succeed in this, technology alone will not be enough. We must also un- and re-learn to grow as individuals and communities.

\hspace{24pt} My learning around Indigenous knowledge systems has revealed to me the importance of shared narratives in the development of mutual understanding, whether that relates to ceremony, to treaties, or to sustainable development in the 21st century. Without a uniting framework of being and becoming, we are fragmented and divided about who we are and what we ought to do - individually, communally, and as a species. I have learned that this is due to the processes of colonization, which have disconnected people from the land and made it nearly impossible to live in place.

\hspace{24pt} If advancements in scientific understanding are to continue to addrress human flourishing, they must be made with a good mind and an understanding of the ways they will impact all human beings and the Mother Earth. To do anything else but this is to condemn future generations to a life worse than our own, and to perpetuate cycles of trauma that have endured for centuries.

\clearpage

% References
\printbibliography[heading=bibintoc]

\end{document}